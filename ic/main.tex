\documentclass[12pt]{ic-final-model}

\begin{document}

\customTitle{Relatório Final - IFSP}

O Relatório Final deverá conter no máximo 25 páginas, em tamanho A4, fonte Arial, corpo 12, 
espaçamento 1,5, contendo:

\begin{enumerate}[label=\alph*)]

    \item Título do relatório

    \item Nome, telefones e e-mail do bolsista;

    \item Nome, endereço, telefone e e-mail da Instituição de vínculo da bolsa – utilizar:

    \begin{quote}\itshape
        Instituto Federal de Educação, Ciência e Tecnologia de São Paulo-IFSP\\[2pt]
        Rua Pedro Vicente, 625 – Canindé – São Paulo-SP\\[2pt]
        CEP:01109-010\\[2pt]
        Telefone: 11-3775-4570\\[2pt]
        e-mail: prp@ifsp.edu.br,
    \end{quote}

    \item Nome, telefones, e-mail do professor orientador;
\end{enumerate}

\vspace{1em}

\noindent
\section{Resumo}

Resumir o projeto de pesquisa.
lorem ipsum \#teste
lorem


\section{Apresentação(Introdução, Justificativa e Objetivos)}

\subsection{Introdução}

Este é um parágrafo normal de texto.  
Você pode quebrar linhas usando uma linha em branco.

Exemplo de citação \cite{vaswani2017attention} \cite{abacate}

\subsection{Justificativa}

Apresentar a justificativa para o projeto de pesquisa.

\subsection{Objetivo}

Apresentar a justificativa para o projeto de pesquisa.

\section{Desenvolvimento (Metodologia e Análise)}

Descrever os métodos utilizados no desenvolvimento do projeto de pesquisa, e análise dos resultados obtidos.

asdadasd

Você pode usar \textbf{negrito}, \textit{itálico} e \underline{sublinhado}.

\section{Conclusão (Resultados da pesquisa)}

Expor os resultados alcançados, em conformidade com o Plano de Trabalho

\section{Referências Bibliográficas}

% Use a BibTeX style and point to the .bib file (without the .bib extension).

\bibliographystyle{abbrv}
\bibliography{referencias}


\vspace{2cm}

\begingroup

\noindent{Assinatura do orientador: \hrulefill}

\vspace{1cm}

\noindent{Assinatura do bolsista: \hrulefill}

\endgroup

\end{document}
